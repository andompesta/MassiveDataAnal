\documentclass[a4paper]{article}
\usepackage[english]{babel}

%selezioniamo un font che abbia gli accenti
\usepackage[T1]{fontenc}
%permette di scrivere le lettere accentate (aumenta il tempo di compilazione)
%\usepackage[latin1]{inputenc}

\usepackage{color}
\usepackage{graphicx}

%link riferimenti incrociati
\usepackage[colorlinks=true,pdftex]{hyperref}

%formule matematiche
\usepackage{amsmath,amsthm,amsfonts,amssymb,mathtools}

%\newtheorem{theorem}{Theorem}[section]
%\newtheorem{lemma}[theorem]{Lemma}
%\newtheorem{esempio}{Esempio}[section]
%\newtheorem{prop}{Proposizione}[section]
%\newtheorem{corollary}[theorem]{Corollary}

%\newcommand{\abs}[1]{\left| #1 \right|} % per valore assoluto
%\newcommand{\R}{\mathbb{R}} 			% per simbolo reali
%\newcommand{\FT}{\xleftrightarrow{FT}} 	% per la TCF
%\newcommand{\rect}[2]{\rectangular \left( \frac{#1}{#2} \right)}

%\DeclareMathOperator*{\sgn}{sgn}
%\DeclareMathOperator*{\sinc}{sinc}
%\DeclareMathOperator*{\rectangular}{rect}
%\DeclareMathOperator*{\grad}{u}
      
%eliminiamo la spaziatura extra alla fine dei periodi
\frenchspacing
%indentazione all'inizio dei capitoli
\usepackage{indentfirst}

%inseriamo 4 livelli di numerazioni (capitoli, sezioni, sottosezioni e sottosottosezioni)
\setcounter{secnumdepth}{3}
%diciamo che nell'indice la profondità deve essere 4 (vd. sopra)
\setcounter{tocdepth}{3}


%%%%%%%%%%%%%%%%%%%%%%%%%%%%%%%%%%%%%%%%%%%%%%%%%%%%%%%%%%%%%%%%%%%%%%%%%%

\begin{document}

%%%%%%%%%%%%%%%%%%%%%%%%%%%%%%%%%%%%%%%%%%%%%%%%%%%%%%%%%%%%%%%%%%%%%%%%%%
%	Title
%%%%%%%%%%%%%%%%%%%%%%%%%%%%%%%%%%%%%%%%%%%%%%%%%%%%%%%%%%%%%%%%%%%%%%%%%%
\title{Massive Data Analytics'\\Project Proposal}
\author{Sandro Cavallari, Marco Giglio, Paolo Morettin}
\date{}
\maketitle

%%%%%%%%%%%%%%%%%%%%%%%%%%%%%%%%%%%%%%%%%%%%%%%%%%%%%%%%%%%%%%%%%%%%%%%%%%
%	Chapters
%%%%%%%%%%%%%%%%%%%%%%%%%%%%%%%%%%%%%%%%%%%%%%%%%%%%%%%%%%%%%%%%%%%%%%%%%%
% - Problem description.
% - Your idea and a sketch of your solution.
% - Proposed plan of work.
% - Datasets you will use for the experimental evaluation.
% - The 3-4 most relevant references.
\section{Introduction}
Social networks experienced an exponential growth in the last five to ten years.
In a few years they became one of the most used communication media:
several people, nowadays, tend to spend many hours per day writing on their
\emph{walls}, \emph{twitting} etc and basically every big company, important
personality, or club, manages accounts on several social networks, using them as
its most important communication media.

Given the increasing role of social networks in everyday's lives, researches
became interested to them, questioning how they affect our privacy and behavior
\cite{Debatin}\cite{Acar} or examining the role they fulfilled during some
important recent events, such as the Arab Spring \cite{Howard}\cite{Lotan}.

\section{Project description}
Our interest is to monitor trends on a social network, understand whether people
are feeling positive or negative toward a certain topic and correlate this
feeling with recent news coming from newspapers. In detail, our project aims in
developing a methodology in order to:
\begin{enumerate}
\item understand when the common feeling about a certain topic shifts from positive
to negative, as done in \cite{Bifet};
\item correlate this shift to news coming from newspapers and news agencies.
\end{enumerate}

The social network we will focus on is Twitter, an online social network that
allows users to upload short text messages (\emph{tweets}) of up to 140 characters.

\section{Work plan}
We can divide our work in 3 different phases: the first one to collect data; the
second one to find sentiment change in tweets; the last one to correlate the
results coming from phase 2 to news coming from other sources.

\subsection{Phase 1}
All the analysis we want to perform are only possible on big datasets; otherwise
it becomes hard to correctly identify a sentiment shift. For this reason, we
need to collect a big amount of data coming both from Twitter and from
newspapers and news agencies. 

We are planning to collect these information for one month, exploiting a web
server up 24 hours per day. In order to collect Twitter data, we will use a Java
library called \emph{twitter4j} whereas the news will be extracted from several
RSS feeds. All the resulting data will be stored on a database on the server.

\subsection{Phase 2}
All the data collected in phase 1 will be download on faster machines to be
analyzed. This analysis can be further subdivided in two steps: in the first one
we need to classify each tweet according to its topic; in the second one we aim
in understanding whether people are feeling positive or negative toward that
topic and in which instant in time they change their opinion.

Both the tasks are maden easier by the short length which characterize tweets
and researchers already tried and succeded in similar tasks (see
		\cite{Wang},\cite{Sriram},\cite{Lee} for topic classification and
		\cite{Bifet},\cite{Go},\cite{Palpanas} for feeling detection), hence
several techniques will be investigated and analyzed in order to find the one
which better suits what we are trying to accomplish.

\subsection{Phase 3}
Once we detect a change in how people are feeling toward a certain topic we want
to find a correlation between this change and recent news. There is not much
literature about this task, hence this is the step on which we shall
probably spend more time and energy.

It is difficult to state now what will be the best approach in order to solve
this problem, but some attempts could be done using some machine learning and
text analysis techniques or analyzing links that might be present in tweets.

In addition, we can expect that only important news cause important changes,
hence we can delete from our dataset news that appear to be read by just a small
amount of people and consider only news which are read by a majority of people.

\section{Conclusion}
In these few pages we proposed some new ideas to solve a problem which has not
yet been addressed by the scientific community: correlate a change in the common
feelings of twitter users toward a certain topic to news coming from newspapers.

We propose an approach in three steps which consists in collecting data,
classify tweets according to their topics and expressed feelings and finally in
using some machine learning and text analysis techniques in order to discern
which news caused a certain change in common feelings.


\begin{thebibliography}{9}
\bibitem{Debatin}
Debatin B. et al., \emph{``Facebook and Online Privacy: Attitudes, Behaviors,
and Unintended Consequences''}, Journal of Computer-Mediated Communication,
15, pg. 83-108 (2009)
\bibitem{Acar}
Acar A., \emph{``Antecedents and Consequences of Online Social Networking
Behavior: The Case of Facebook''}, Journal of Website Promotion Vol. 3, N.
1-2, pg. 62-83 (2008)
\bibitem{Howard}
Howard P. et al., \emph{``Opening Closed Regimes: What Was the Role of Social Media
During the Arab Spring?''}, ICT4D Bibliography (2011)
\bibitem{Lotan}
Lotan G. et al., \emph{``The Revolutions Were Tweeted: Information Flows During
the 2011 Tunisian and Egyptian Revolutions''}, International Journal of
Communication 5 (2011)
\bibitem{Bifet}
Albert B. et al., \emph{``Detecting Sentiment Change in Twitter Streaming
	Data''}, JMLR: Workshop and Conference Proceedings 17 (2011) 5-11
\bibitem{Go}
Go A., Bhayani R., Huang, L., \emph{``Twitter sentiment classification using
	distant supervision''}, CS224N Project Report, Stanford, 1-12 (2009)
\bibitem{Palpanas}
Tsytsarau M., Palpanas T., \emph{``Survey on mining subjective data on the
	web''}, Data Mining and Knowledge Discovery, Vol. 24, N. 3, pg. 478-514,
	Springer (2012)
\bibitem{Wang}
Wang X. et al., \emph{``Topic sentiment analysis in twitter: a graph-based
hashtag sentiment classification approach''}, 20th ACM International
Conference on Information and Knowledge Management (2011) 
\bibitem{Sriram}
Sriram B. et al., \emph{``Short Text Classification in Twitter to Improve
	Information Filtering''}, Proceedings of the 33rd International ACM SIGIR
	Conference on Research and Development in Information Retrieval, pg. 841-842 (2010)
\bibitem{Lee}
Lee K. et al., \emph{``Twitter Trending Topic Classification''}, IEEE 11th
International COnference on Data Mining Workshops (2011)
\end{thebibliography}


\end{document}
