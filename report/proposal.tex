\documentclass[a4paper]{article}
\usepackage[english]{babel}

%selezioniamo un font che abbia gli accenti
\usepackage[T1]{fontenc}
%permette di scrivere le lettere accentate (aumenta il tempo di compilazione)
%\usepackage[latin1]{inputenc}

\usepackage{color}
\usepackage{graphicx}

%link riferimenti incrociati
\usepackage[colorlinks=true,pdftex]{hyperref}

%formule matematiche
\usepackage{amsmath,amsthm,amsfonts,amssymb,mathtools}

%\newtheorem{theorem}{Theorem}[section]
%\newtheorem{lemma}[theorem]{Lemma}
%\newtheorem{esempio}{Esempio}[section]
%\newtheorem{prop}{Proposizione}[section]
%\newtheorem{corollary}[theorem]{Corollary}

%\newcommand{\abs}[1]{\left| #1 \right|} % per valore assoluto
%\newcommand{\R}{\mathbb{R}} 			% per simbolo reali
%\newcommand{\FT}{\xleftrightarrow{FT}} 	% per la TCF
%\newcommand{\rect}[2]{\rectangular \left( \frac{#1}{#2} \right)}

%\DeclareMathOperator*{\sgn}{sgn}
%\DeclareMathOperator*{\sinc}{sinc}
%\DeclareMathOperator*{\rectangular}{rect}
%\DeclareMathOperator*{\grad}{u}
      
%eliminiamo la spaziatura extra alla fine dei periodi
\frenchspacing
%indentazione all'inizio dei capitoli
\usepackage{indentfirst}

%inseriamo 4 livelli di numerazioni (capitoli, sezioni, sottosezioni e sottosottosezioni)
\setcounter{secnumdepth}{3}
%diciamo che nell'indice la profondità deve essere 4 (vd. sopra)
\setcounter{tocdepth}{3}


%%%%%%%%%%%%%%%%%%%%%%%%%%%%%%%%%%%%%%%%%%%%%%%%%%%%%%%%%%%%%%%%%%%%%%%%%%

\begin{document}

%%%%%%%%%%%%%%%%%%%%%%%%%%%%%%%%%%%%%%%%%%%%%%%%%%%%%%%%%%%%%%%%%%%%%%%%%%
%	Title
%%%%%%%%%%%%%%%%%%%%%%%%%%%%%%%%%%%%%%%%%%%%%%%%%%%%%%%%%%%%%%%%%%%%%%%%%%
\title{Massive Data Analytics'\\Project Proposal}
\author{Sandro Cavallari, Marco Giglio, Paolo Morettin}
\maketitle

%%%%%%%%%%%%%%%%%%%%%%%%%%%%%%%%%%%%%%%%%%%%%%%%%%%%%%%%%%%%%%%%%%%%%%%%%%
%	Chapters
%%%%%%%%%%%%%%%%%%%%%%%%%%%%%%%%%%%%%%%%%%%%%%%%%%%%%%%%%%%%%%%%%%%%%%%%%%
% - Problem description.
% - Your idea and a sketch of your solution.
% - Proposed plan of work.
% - Datasets you will use for the experimental evaluation.
% - The 3-4 most relevant references.
\section{Introduction}
Social networks experienced an exponential growth in the last five to ten years.
In a few years they became one of the most used communication media:
several people, nowadays, tend to spend many hours per day writing on their
\emph{walls}, \emph{twitting} etc and basically every big company, important
personality, or club, manages accounts on several social networks, using them as
its most important communication media.

Given the increasing role of social networks in everyday's lives, researches
became interested to them, questioning how they affect our privacy and behavior
\cite{Debatin}\cite{Acar} or examining the role they fulfilled during some
important recent events, such as the Arab Spring \cite{Howard}\cite{Lotan}.

\section{Project description}
Our interest is to monitor trends on a social network, understand whether people
are feeling positive or negative toward a certain topic and correlate this
feeling with recent news coming from newspapers. In detail, our project aims in
developing a methodology in order to:
\begin{enumerate}
\item understand when the common feeling about a certain topic shifts from positive
to negative;
\item correlate this shift to news coming from newspapers and news agencies.
\end{enumerate}

The social network we will focus on is Twitter, an online social network that
allows users to upload short text messages (\emph{tweets}) of up to 140 characters.

\subsection{Work plan}
In order to perform our analysis we need a big dataset containing tweets and
another one containing the news published by popular newspapers and news
agencies during the same temporal interval. Our team is planning in renting a
server having low computational power but high bandwidth and availability. The
server will be responsible for collecting both tweets and news 24 hours per day
for some weeks. All the information collected by the server will be stored on a
database.

In a second step, all the information will be downloaded on faster machines and
analysed in order to discover trends and correlations. 

\subsection{Dataset}

As over-mentioned the Dataset for this work will be composed by:

\begin{enumerate}
	\item Twitt downloaded by a server using twitter4j and saved on a MongoDb
	\item News collected by the Rss Feed of the most important newspaper websites
\end{enumerate}

Our plan is to download at least 1 month of data and only after start to analyse the correlations.

\begin{thebibliography}{9}
\bibitem{Debatin}
Debatin B. et al., \emph{``Facebook and Online Privacy: Attitudes, Behaviors,
and Unintended Consequences''}, Journal of Computer-Mediated Communication,
15, pg. 83-108 (2009)
\bibitem{Acar}
Acar A., \emph{``Antecedents and Consequences of Online Social Networking
Behavior: The Case of Facebook''}, Journal of Website Promotion Vol. 3, N.
1-2, pg. 62-83 (2008)
\bibitem{Howard}
Howard P. et al., \emph{``Opening Closed Regimes: What Was the Role of Social Media
During the Arab Spring?''}, ICT4D Bibliography (2011)
\bibitem{Lotan}
Lotan G. et al., \emph{``The Revolutions Were Tweeted: Information Flows During
the 2011 Tunisian and Egyptian Revolutions''}, International Journal of
Communication 5 (2011)
\end{thebibliography}


\end{document}
