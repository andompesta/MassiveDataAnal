\documentclass[a4paper]{article}
\usepackage[english]{babel}

%selezioniamo un font che abbia gli accenti
\usepackage[T1]{fontenc}
%permette di scrivere le lettere accentate (aumenta il tempo di compilazione)
%\usepackage[latin1]{inputenc}

\usepackage{color}
\usepackage{graphicx}

%link riferimenti incrociati
\usepackage[colorlinks=true,pdftex]{hyperref}

%formule matematiche
\usepackage{amsmath,amsthm,amsfonts,amssymb,mathtools}

%\newtheorem{theorem}{Theorem}[section]
%\newtheorem{lemma}[theorem]{Lemma}
%\newtheorem{esempio}{Esempio}[section]
%\newtheorem{prop}{Proposizione}[section]
%\newtheorem{corollary}[theorem]{Corollary}

%\newcommand{\abs}[1]{\left| #1 \right|} % per valore assoluto
%\newcommand{\R}{\mathbb{R}} 			% per simbolo reali
%\newcommand{\FT}{\xleftrightarrow{FT}} 	% per la TCF
%\newcommand{\rect}[2]{\rectangular \left( \frac{#1}{#2} \right)}

%\DeclareMathOperator*{\sgn}{sgn}
%\DeclareMathOperator*{\sinc}{sinc}
%\DeclareMathOperator*{\rectangular}{rect}
%\DeclareMathOperator*{\grad}{u}
      
%eliminiamo la spaziatura extra alla fine dei periodi
\frenchspacing
%indentazione all'inizio dei capitoli
\usepackage{indentfirst}

%inseriamo 4 livelli di numerazioni (capitoli, sezioni, sottosezioni e sottosottosezioni)
\setcounter{secnumdepth}{3}
%diciamo che nell'indice la profondità deve essere 4 (vd. sopra)
\setcounter{tocdepth}{3}


%%%%%%%%%%%%%%%%%%%%%%%%%%%%%%%%%%%%%%%%%%%%%%%%%%%%%%%%%%%%%%%%%%%%%%%%%%


\begin{document}

%%%%%%%%%%%%%%%%%%%%%%%%%%%%%%%%%%%%%%%%%%%%%%%%%%%%%%%%%%%%%%%%%%%%%%%%%%
%	Title
%%%%%%%%%%%%%%%%%%%%%%%%%%%%%%%%%%%%%%%%%%%%%%%%%%%%%%%%%%%%%%%%%%%%%%%%%%
\title{Massive Data Analytics'\\Project Proposal}
\author{Sandro Cavallari, Marco Giglio, Paolo Morettin}
\maketitle

%%%%%%%%%%%%%%%%%%%%%%%%%%%%%%%%%%%%%%%%%%%%%%%%%%%%%%%%%%%%%%%%%%%%%%%%%%
%	Chapters
%%%%%%%%%%%%%%%%%%%%%%%%%%%%%%%%%%%%%%%%%%%%%%%%%%%%%%%%%%%%%%%%%%%%%%%%%%
% - Problem description.
% - Your idea and a sketch of your solution.
% - Proposed plan of work.
% - Datasets you will use for the experimental evaluation.
% - The 3-4 most relevant references.
\section{Problem description}
Social networks experienced an exponential growth in the last 5 to ten years.
Several people, nowadays, tend to spend many hours per day writing on their
\emph{walls}, \emph{twitting} etc, and social networks are often be proved to be
modern news aggregator.

Given that, it is important to monitor the most popular trends on social
networks and to understand whether people are feeling positive or negative
toward a certain topic. Our project aims in developing a methodology in order
to:
\begin{enumerate}
\item understand when the common feeling about a certain topic shifted from positive
to negative;
\item correlate this shift to news coming from newspapers and news agencies.
\end{enumerate}

\section{Work plan}
In order to perform our analysis we need a big dataset containing tweets and
another one containing the news published by popular newspapers and news
agencies during the same temporal interval. Our team is planning in renting a
server having low computational power but high bandwidth and availability. The
server will be responsible for collecting both tweets and news 24 hours per day
for some weeks. All the information collected by the server will be stored on a
database.

In a second step, all the information will be downloaded on faster machines and analysed in order to discover trends and correlations.

\section{Dataset}

As over-mentioned the Dataset for this work will be composed by:

\begin{itemize}
	\item Twitt downloaded by a server using twitter4j and saved on a MongoDb
	\item News collected by the Rss Feed of the most important newspaper websites
\end{itemize}

Our plan is to download at least 1 month of data and only after start to analyse the correlations.

\begin{thebibliography}{9}
\bibitem{}
Text
\end{thebibliography}


\end{document}
