\documentclass{acm_proc_article-sp-sigmod07}

\usepackage{times}
\usepackage{graphicx}
\usepackage{subfigure}
\usepackage{color}
\usepackage{url}
\usepackage{mathtools}
\usepackage{amsfonts}
\usepackage{newclude}
\usepackage{float}
\usepackage{listings}
\usepackage{caption}
\usepackage{rotating}
\usepackage{footnote}
%\usepackage[toc,page]{appendix}

%\usepackage{hyperref}
% Pacchetti per poter scrivere con la tastiera italiana
\usepackage{ucs}
\usepackage[utf8x]{inputenc}
\usepackage[T1]{fontenc} %Aumenta la resa dei caratteri quando si stampa

% Pacchetto per scrivere in italiano
\usepackage[english]{babel}

% Pacchetti matematici vari
% Pacchetti base

\usepackage{makeidx} % Pacchetto per l'indice analitico
\usepackage{mparhack} % Pacchetto che corregge alcuni errori nei magini della pagina 
\usepackage{marginnote} % Pacchetto che permette di scrivere note a margine 
\usepackage{listings} % Pacchetto necessario per scrivere codice sorgente 
\usepackage{braket} % Pacchetto che permette di scrivere tutte le parentesi
\usepackage{csquotes}
\usepackage{setspace}
\usepackage{graphicx}
\usepackage{tabularx}
\usepackage{amsmath}
\usepackage{changepage}
\usepackage{cite}
\usepackage{algorithmic} % Pacchetto per pseudocodice
\usepackage[english]{varioref}

\newcommand\footnoteref[1]{\protected@xdef\@thefnmark{\ref{#1}}\@footnotemark}

\usepackage[letterpaper]{geometry}
\geometry{top=1.0in, bottom=1.0in, left=1.0in, right=1.0in}

\lstdefinestyle{customc}{
  belowcaptionskip=1\baselineskip,
  breaklines=true,
  frame=L,
  xleftmargin=\parindent,
  language=C,
  showstringspaces=false,
  basicstyle=\footnotesize\ttfamily,
  keywordstyle=\bfseries\color{blue},
  commentstyle=\itshape\color{green},
  identifierstyle=\color{black},
  stringstyle=\color{orange},
}

\lstdefinestyle{customasm}{
  belowcaptionskip=1\baselineskip,
  frame=L,
  xleftmargin=\parindent,
  language=[x86masm]Assembler,
  basicstyle=\footnotesize\ttfamily,
  commentstyle=\itshape\color{purple!40!black},
}

\lstset{escapechar=@,style=customc}


\begin{document}
\title{Event summarization from news and tweets correlation}
\author{Cavallari Sandro\\Giglio Marco\\Morettin Paolo
\thanks{We would like to thank Dr. Mikalai Tsytsarau for his valuable advices and for sharing with us his datasets and tools.}}

%
\maketitle
\begin{abstract}
In this report we are evaluating the task of extracting a summary of the main event that caused a shift on the opinion of Twitter users. 
This paper presents several techniques, which can be used to analyze these \emph{sentiment shift} and to find the events that might have caused them.
For each technique presented below we are inspecting some major aspects, which are important in the context of a data mining application, such as scalability and efficiency.
 A comparison of the different techniques is presented as well.
 \end{abstract}   

\section{Introduction}
This work aims to find a summary of the events that might have caused a sentiment shift.

While event extraction in Natural Language Processing (NLP) techniques are mature, their performance on tweets inevitably degrades due to the inherent dispersion in short texts. Since tweets contain heterogeneous language structures and are at most 140 characters long, to identify and extract events from tweets is quite tricky: instead linking single tweets to news articles allows us to extract data from well structured and longer text.


%Due to the fact that our dataset is composed by labelled tweets, we manage to extract news from the New York Times archive basing the research on tweet's label. Starting form a good Tweet-News classification, allowed us to correlate tweet and news that belong to the same macro-topics avoiding to correlate object of different arguments.

Moreover NLP document summarization techniques usually exploit language dependent methods, whereas we want a methodology as language independent as possible.

This paper is structured as follows: in the next chapter we present some works
which are somehow related with the problem we are addressing, then we present a more
formal definition of our problem and some techniques which might be
used to solve it. The techniques which will be later discussed are:
\begin{itemize}
	\item a technique exploiting the \emph{SpaceSaving} algorithm presented in
		\cite{SS};
	\item a technique exploiting \emph{Latent Semantic Indexing} \cite{LSA};
	\item the third one based on the popular \emph{Tf/Idf};
	\item the last (but not least) which take advantage of ngram graphs
		\cite{Ngram}
\end{itemize}
We will discuss each of these techniques identifying their main advantages and disadvantages paying a particular attention to their efficiency and scalability issues.
At the end of the paper we will present some conclusions and draw some suggestions on how this work may be extended.

\section{Related Work}
\include*{chapters/RelatedWork}

\section{Problem Definition}
\include*{chapters/ProblemDefinition}

\section{Proposed Approach}
\include*{chapters/ProposedApproach}

\section{Experimental Evaluation}
\include*{chapters/ExperimentalEvaluation}

\section{Conclusions}
\include*{chapters/Conclusions}

\begin{thebibliography}{9}
\bibitem{LSA} 
	DEERWESTER, Scott C.. , et al. ``\emph{Indexing by latent semantic analysis}''. JASIS, 1990, 41.6: 391-407.
\bibitem{LSA2} 
	LANDAUER, Thomas K., et al. (ed.). ``\emph{Handbook of latent semantic analysis}''. Psychology Press, 2013.
\bibitem{Gensim}
	ŘEHŮŘEK, Radim, et al. ``\emph{Software framework for topic modelling with large corpora}''. 2010.
\bibitem{Ngram}
	GIANNAKOPOULOS, George, et al. ``\emph{Automatic Summarization from Multiple Documents : N-Gram Graph}''. 2009.
\bibitem{LTN}
	GUO, Weiwei, et al. ``\emph{Linking Tweets to News: A Framework to Enrich Short Text Data in Social Media}''. In: ACL (1). 2013. p. 239-249.
\bibitem{WTMF}
	GUO, Weiwei; DIAB, Mona. ``\emph{Modeling sentences in the latent space}''. In: Proceedings of the 50th Annual Meeting of the Association for Computational Linguistics: Long Papers-Volume 1. Association for Computational Linguistics, 2012. p. 864-872.
\bibitem{MSC}
	FILIPPOVA, Katja. ``\emph{Multi-sentence compression: Finding shortest paths in word graphs}''. In: Proceedings of the 23rd International Conference on Computational Linguistics. Association for Computational Linguistics, 2010. p. 322-330.
\bibitem{Bifet}
	BIFET, Albert, et al. ``\emph{Detecting Sentiment Change in Twitter Streaming Data}''. WAPA, 2011
\bibitem{SS}
	METWALLY Ahmed, et al. ``\emph{Efficient Computation of Frequent and Top-k
	Elements in Data Streams}'', Lecture Notes in Computer Science Volume 3363, 2005, pp 398-412
\bibitem{nltk}
	BIRD, Steven. ``\emph{NLTK: The Natural Language Toolkit}''.Proceedings of the COLING/ACL '06 on Interactive presentation sessions
Pages 69-72. 2006

\end{thebibliography}
\end{document}
