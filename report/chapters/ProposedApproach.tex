In this section we present different approaches which might be used to solve the problem we are addressing. Of course, each approach has advantages and disadvantages which we will list while showing them.

%All the approaches have some common preprocessing phase that are used for produce and clean the data intermediate. So we assume that the following operations are prerequisite for a good computation of all the methodologies. 
%Initially we used a framework form analyse the different tweet time series; this tool allowed us to detect all the sentiment shift points (we call that point \textbf{Contradiction Point}) and save the contradiction time windows in .json files. 
%With the contradiction point time windows we were able to extract all the tweet that cause the shift and save them in other .json files, so we were able to read the contradiction tweet every time we need it.

\subsection*{SpaceSaving approach}
\include*{chapters/SS}

\subsection*{LSI}
\include*{chapters/LSI}

\subsection*{N-Gram Graph}
\include*{chapters/Ngram}
