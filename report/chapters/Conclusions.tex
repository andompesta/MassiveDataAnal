During the development of this work we found out that working in a blurred environments as the one of the natural language is quite difficult.
Many problem were due to the comparison of the different approach: every algorithm have different output format and normalizing the format in a standard form it's quite challenging. 
Developing an automatic validation system for the results will help a lot the comparison and may produce more objective results.

\subsection*{Future improvements}
\subsubsection*{News retrieval}
One of the biggest difficulties we had to face is the lack of news for the
selected time period (i.e. year 2009). Unfortunately, retrieving 4 years old
news is not as easy as one may think and many news agencies and newspapers only
provide this kind of service with a payed subscription.

Moreover, we encountered additional issues when dealing with scientific or
entertainment topics. Often, only a few lines are dedicated to this kind of
topics in newspapers and they rarely reflect people opinions.

To perform a deeper analysis we think it would be necessary to focus on more and
more differentiated news sources, looking also for scientific newspapers (e.g.
\emph{Wired}) and entertainment ones (e.g. \emph{The Hollywood reporter}).

\subsubsection*{Redundancy removal}
NGG is a technique developed primary with the goal of \emph{inter-summary} generation: a summarization process that takes into account information already available to the reader. In order to achieve this goal, in a good summary every new sentence must add as less redundant information as possible.
According to the proposal of the NGG developer, implementing the following function can improve the quality of the summary created:
\begin{itemize}
	\item Starting form the first sentence of the summary, compute his NGG representation $G_{sum}$
	\item Remote $G_{sum}$ form the intersection of all the news to summarize ($C_{u}$). The new graph $G_{sum}^{\prime}$ = $G_{sum} \triangle G_{in}$
	\item For every candidate sentence of the news that has not been already used:
	\begin{itemize}
		\item extract its n-gram graph representation $G_{cs}$
		\item keep only $G_{cs}^{\prime} = G_{cs} \triangle C_{in}$, because we want to add sentence with low redundancy
		\item Computing the NVS between $G_{cs}^{\prime}$ and $G_{sum}^{\prime}$ give a \emph{redundancy score}
	\end{itemize}
	\item rank the candidate sentence using as score their NVS value respect $G_{in}$ decreased by the redundancy score
\end{itemize}
this functionality was developed by the author of NGG tool and was used for inter-summary generation. Unfortunately for time reason we weren't able to test his performance in our research, but we think that will produce much more human readable summary with more information.

\subsubsection*{Adaptive time windows}
All the methodologies compute the correlation between news and tweet using a fixed-length time windows. This approach is quite simple and can produce good performance since there are a good number of news in the selected time slice. 

In order to face tricky situation where there are only few news for the selected interval, a more sophisticated approach should adapt the length of the time windows according to the number of news available:
precisely, more news are present in the time period of the sentiment shift and shorter should be the amount of time added to the time windows of CP.
Developing and testing an adaptive time windows can increase the performances of the all the methodologies and is one of the main point for the future improvements.


