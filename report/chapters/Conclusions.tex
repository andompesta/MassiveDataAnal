The main goal of our research work is to link tweets to news articles to extract relevant information from news 
instead of only tweets since the later source can be used as a wider and more structured source of information. 
In fact the main focus of the project is to create a correlation between tweets
regarding a well known topic and 
the most relevant news that are dealing with the same topic in a defined time window.
 
All approaches used to address this topic proved to be valid and to return congruent results. 
Of course the quality of the result produced by the various approaches is different and each of them
might be useful in different contexts: TF-IDF and Space Saving are still less
accurate than NGG, but can be used in environments where 
real time data feeds must be processed; the N-Gram Graph is slightly
more accurate. LSI probably needs more refinements.


It is quite obvious that NGG approach could be used in a much wider scenario where tweets (as a measure of 
human behaviour) could be related to specific news and vice versa. Of course to achieve this result there 
are more complex problems that must be better described and solved.

Our work clearly identify at least three major new research directions:
\begin{enumerate}
	\item automatically computing the proper window frame depending on the topic,
		rather than using a fixed window as in our experiments;
	\item identifying the noise created by news concerning the topic, but not the
		event tweets are focusing on. This is a very challenging research topic especially if the correlation calculation should not use semantic evaluation of text sources to be language and culture  independent 
	\item identifying whether a correlation that links tweets to news is recursively linked to other correlations to cross detect which topic might influence or be influenced by other ones
\end{enumerate}

%\subsection*{Research challenges and problems solved}
%During the development of this work we found out that working in a blurred
%environments as the one of the natural language is quite difficult. One of the
%biggest problem has been to find out how to present different methodologies,
%while making all of them comparable to each other.
%
%Also, the lack of an automatic validation system gives many problems, since the
%results can be interpreted by anyone in a different way.

\subsection*{Future improvements}
\subsubsection*{News retrieval}
One of the biggest difficulties we had to face is the lack of news for the
selected time period (i.e. year 2009). Unfortunately, retrieving 5 years old
news is not as easy as one may think and many news agencies and newspapers only
provide this kind of service with a payed subscription.

Moreover, we encountered additional issues when dealing with scientific or
entertainment topics. Often, only a few lines are dedicated to this kind of
topics in newspapers and they rarely reflect people opinions.

To perform a deeper analysis we think it would be necessary to focus on more and
more differentiated news sources, looking also for scientific newspapers (e.g.
\emph{Wired}) and entertainment ones (e.g. \emph{The Hollywood reporter}).

\subsubsection*{Redundancy removal}
NGG is a technique developed primary with the goal of \emph{inter-summary} generation: a summarization process that takes into account information already available to the reader. In order to achieve this goal, in a good summary every new sentence must add as less redundant information as possible.
According to the proposal of the NGG developer, implementing the following function can improve the quality of the summary created:
\begin{itemize}
	\item Starting form the first sentence of the summary, compute his NGG representation $G_{sum}$
	\item Remote $G_{sum}$ form the intersection of all the news to summarize ($C_{u}$). The new graph $G_{sum}^{\prime}$ = $G_{sum} \triangle G_{in}$
	\item For every candidate sentence of the news that has not been already used:
	\begin{itemize}
		\item extract its n-gram graph representation $G_{cs}$
		\item keep only $G_{cs}^{\prime} = G_{cs} \triangle C_{in}$, because we want to add sentence with low redundancy
		\item Computing the NVS between $G_{cs}^{\prime}$ and $G_{sum}^{\prime}$ give a \emph{redundancy score}
	\end{itemize}
	\item rank the candidate sentence using as score their NVS value respect $G_{in}$ decreased by the redundancy score
\end{itemize}
this functionality was developed by the author of NGG tool and was used for inter-summary generation. Unfortunately for time reason we weren't able to test his performance in our research, but we think that will produce much more human readable summary with more information.

\subsubsection*{Adaptive time windows}
All the methodologies compute the correlation between news and tweet using a
fixed-length time window. This approach is indeed simple and we have seen that,
still, it allows to obtain good outputs. 

However, tricky situation may require a more sophisticated approach: if only a
few news are available or in event characterized by big variance, an
automatically adjusting window might be a relevant factor to obtain good
results.

Developing an adaptive time window can increase the performances of all the methodologies and is one of the main point for the future improvements.

\subsubsection*{Evaluation}
Due to the nature of the output format returned by our tools, presently, we have
no method to automatically compare the results achieved. We presented and
commented some results and we also showed some results achieved by performing a
human evaluation, but indeed this is but a starting point.

In order to have a more unbiased and precise evaluation more experiments on a
bigger set of contradiction points are needed, as well as a severe user
evaluation phase to discern which methodology behaves better on which
contradiction point and why.

Moreover, the implementation and exploitation of an automatic tool might be
possible.

