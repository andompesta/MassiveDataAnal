A comparison of the proposed methodologies is presented in table
\ref{tab:methComp}.

The SpaceSaving approach is, indeed, the simplest but it can be applied in real
time and is very scalable with respect to the number of tweets and the number of
words contained in the tweets.

LSI and TF-IDF have similar characteristics, providing background removal. Both
of them might be applied in real time, but they may suffer if the number of
different terms in the corpus of tweets is big.

The Ngram Graphs method is the only one not based on the \emph{bag of words}
approach. It is probably the most complex approach presented, providing
background removal and considering word order in its analysis, but it is the most
expendive in terms of computational time and memory, making it absolutely unfeasible for
real time applications and unpractical when the number of tweets or news in the
contradiction window is big.

\begin{table}[htb]
	\centering
	\begin{tabular}{r|c|c|c|c}
		& \begin{sideways}Words order\end{sideways} & \begin{sideways}Background
			removal\end{sideways} & \begin{sideways}Real Time\end{sideways} &
				\begin{sideways}Scalability\end{sideways} \\
		\hline
		SpaceSaving & No & No & Yes & High \\
		LSI & No & Yes & Maybe & Medium \\
		TF-IDF & No & Yes & Maybe & Medium\\
		NGram Graphs & Yes & Yes & No & Low\\
		\end{tabular}
	\caption{Comparison of proposed methods}
	\label{tab:methComp}
\end{table}

