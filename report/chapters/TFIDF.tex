We also tried an approach based on TF-IDF to correlate the sentiment shifts with the news. TF-IDF is a widely used statistic in NLP and is indeed a effective technique to extract keywords from documents. Our methodology works as follows:
\begin{itemize}
\item For each sentiment shift, its tweets are merged in a single document, in order to be filtered and tokenized. Using all the tweets of the given topic as corpus, a first list of TF-IDF weighted tokens is computed. Then, another list for the same document is computed, this time using as corpus just the shift documents. Finally the two are merged, summing the TF-IDF values of the common elements, and the best $k$ keywords according to their TF-IDF score are stored.
\item The set of candidate news to be tested is chosen according to a window. As for the shift documents, for each news two lists are computed. The first, using as corpus the whole dataset of news of the topic, the other using just the news in the candidate set. This is done to remove the common terms in the topic, and also to have a clear distinction between each candidate. The final result is again a single keywords list for each document.
\item Once we have a keywords list of the sentiment shift and a set of keywords list representing the news, a score is computed using the Jaccard similarity. This allow us to select the news that shares more keywords with the shift.
\end{itemize}
