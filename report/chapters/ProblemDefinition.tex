We define \emph{sentiment} the measurement of feelings performed using specific
tools analyzing the stream of tweets. This kind of measurement is usually done
using some supervised machine learning techniques together with NLP techniques.

Tools exist to track the \emph{sentiment} toward a certain topic and to identify
changes in it. In particular we can consider three major events in a sentiment
timeline:
\begin{enumerate}
	\item there is a change of sentiment from positive to negative (or
		viceversa)
	\item there is no change of sentiment, but there is a peek in the graph
		describing it or its value crosses the average value
	\item there is no change but in the volume of tweets/second
\end{enumerate}
We define the first of these events as \emph{sentiment shift} and the temporal
window in which it happens is called \emph{contradiction point} or
\emph{contradiction window}.

In addition we define \emph{contradiction tweets} and \emph{contradiction news}
respectively the tweets and news published inside a contradiction window and
\emph{background tweets} and \emph{background news} the set of all tweets and
news regarding the addressed topic.

Given these definitions, the problem we aim to solve is to find, among the
contradiction news, the ones which are the more representative of the
contradiction tweets and to present to the user a summary which describes them.
