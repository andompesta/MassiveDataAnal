\subsubsection*{Description of the algorithm}
\emph{SpaceSaving} \cite{SS} is an algorithm which was first presented by Metwally et al.
in 2005 to efficiently compute the most frequent terms in a data stream. It
allows the user to find $k$ words which are among the most frequent in the given
stream. Although this is a heuristic algorithm whose result's correctness is not
guaranteed, SpaceSaving is able to specify the upper bound of the error for each
word presented as output.

The classic implementation is based on a fixed size list of tuple 
\begin{displaymath}
	(word, occurrencies, error)
\end{displaymath}
The stream is read word by word and at any time one of the following condition
is verified:
\begin{enumerate}
	\item \label{SS-among-frequent}
		the word is already in the list of frequent terms
	\item \label{SS-list-not-full}
		the word is not among the frequent terms, but the list is not full
	\item \label{SS-list-full}
		the word is not among the frequent terms and the list is full
\end{enumerate}
If the first condition is verified, then the algorithm will proceed incrementing
the number of occurrencies of that term; if condition \ref{SS-list-not-full} is
verified, instead, we add that word to the list of frequent terms with number of
occurrencies 1 and number of errors 0; if condition \ref{SS-list-full} is
verified, then we scan the list for the term with the lowest number of
occurrencies and replace it with the new word. When this replacement takes place
we set the number of errors equal to the number of occurrencies of the word we
just replaced, then we increment the number of occurrencies.

Our python implementation of this algorithm is slightly different from the
classic one described above, since we test the word against a list of stop words
and we added a hashmap to avoid useless
replacement. In our implementation, then, the following steps are performed:
\begin{enumerate}
	\item if the word is in the stop word lists, does nothing
	\item otherwise compute the hash of the given word
	\item increment the value associated to that hash
	\item check whether the value stored in the hash map is above the number of
		occurrencies of the less frequent term tracked by the SpaceSaving
		algorithm. If so, the replacement occurs, otherwise nothing happens.
\end{enumerate}
The output of this algorithm is a couple $(word, value)$ where the value is
computed as 
\begin{displaymath}
	value = occurrencies - error
\end{displaymath}
In order to reduce the noise we chose to use only couples where the value is
above a certain threshold.

\subsubsection*{Description of the methodology}
We have seen how we exploit the SpaceSaving algorithm to obtain the list of
frequent terms in a text and their
values. 

To accomplish our task, we ran the program described above on the set of tweets inside a
\emph{contradiction point}, thus obtaining the list of frequent terms within the
contradiction; then, we read all the news which were published in the same time
period (a certain error between the publishing date and the contradiction
window) and for each of them we compute its score as
\begin{displaymath}
	newsScore = \sum_{w \in W} value(w)
\end{displaymath}
where $W$ is the list of words in the news and the value is null if that term is
not in the \emph{frequent term list}.

The news with the highest score is selected as the one causing the shift and its
first words are presented as output to the user.

This approach revealed to be very fast and from its formulation we can see that
it can be applied incrementally and in real time, succeding where many other fail. As a drawback, we must say that it does not
take in account the background coming from the topic, hence it is not able to
discern whether a given frequent term is informative or not (e.g. running this
program on the Obama topic it is likely that the most
frequent terms will be ``President'', `Barack'', ``Obama'', ``USA'', hence a
news containing more repetitions of these words will have a high score, even if
those words are not informative at all)
