\subsection*{Experimental setup}
In order to perform the experimental evaluation we used a tool to automatically
detect sentiment shift in tweets coming from year 2009 and regarding several
topics. We also downloaded news from the \emph{New York Times} and \emph{ABC
Australia} on the same topic and spanning on the same period of the tweets.

It's important to remember that tweet have different language structure that need to be normalized, so for cleaning purpose, the following operations were performed on the tweet text before starting the computation of all the different methodologies:
\begin{itemize}
	\item URLs removal from tweets using a regular expression
	\item conversion from Unicode to ASCII
\end{itemize}

After that, we manually labelled each contradiction point with the event which
caused it. The topics, contradiction points and events used for the experiments
are showed in table \ref{tab:setup}

\begin{table*}
	\centering
	\begin{tabularx}{\textwidth}{|l|l|X|}
	\hline
	Contr. point 	& Contr. window 			& Event \\
	\hline 
	Cern1			& 2009-10-29 - 2009-11-08 	& On November 3rd a bird drops a
piece of bread which cause LHC overheating. \\
	Cern2			& 2009-11-24 - 2009-12-04	& On November 30th LHC accelerates protons to an
energy of 1.18 TeV, becoming the world most powerful energy particle
accelerator\\
	FortHood 		& 2009-11-02 - 2009-11-06	& On November 5th a US marine
kills 13 people\\
	HangOver		& 2009-06-21 - 2009-06-27	& undetermined or movie released
	on June 5th\\
	Lcross			& 2009-10-31 - 2009-11-08	& Preliminary findings from
	Lcross. Others announced\\
	Jackson1		& 2009-06-19 - 2009-06-25	& On June 25th Michael Jackson
dies\\
	Jackson2		& 2009-08-23 - 2009-08-29	& On august 28th another popular
musician, Adam Goldstein, dies. The day after, August 29th is Jackson's
birthday\\
	SwineFlu		& 2009-06-19 - 2009-06-23	& The swine flu is recognized as
	a pandemic\\
	\hline
	\end{tabularx}
	\caption{Contradiction points used for experimental evaluation}
	\label{tab:setup}
\end{table*}

\subsection*{Experiments}
\subsubsection*{SpaceSaving evaluation}
The results achieved using the \emph{SpaceSaving} methodology with a maximum
error of three days on the contradiction window are shown in table
\ref{tab:resultsSS}.
\subsubsection*{Description of the algorithm}
\emph{SpaceSaving} \cite{SS} is an algorithm which was first presented by Metwally et al.
in 2005 to efficiently compute the most frequent terms in a data stream. It
allows the user to find $k$ words which are among the most frequent in the given
stream. Although this is a heuristic algorithm whose result's correctness is not
guaranteed, SpaceSaving is able to specify the upper bound of the error for each
word presented as output.

The classic implementation is based on a fixed size list of tuple 
\begin{displaymath}
	(word, occurrencies, error)
\end{displaymath}
The stream is read word by word and at any time one of the following condition
is verified:
\begin{enumerate}
	\item \label{SS-among-frequent}
		the word is already in the list of frequent terms
	\item \label{SS-list-not-full}
		the word is not among the frequent terms, but the list is not full
	\item \label{SS-list-full}
		the word is not among the frequent terms and the list is full
\end{enumerate}
If the first condition is verified, then the algorithm will proceed incrementing
the number of occurrencies of that term; if condition \ref{SS-list-not-full} is
verified, instead, we add that word to the list of frequent terms with number of
occurrencies 1 and number of errors 0; if condition \ref{SS-list-full} is
verified, then we scan the list for the term with the lowest number of
occurrencies and replace it with the new word. When this replacement takes place
we set the number of errors equal to the number of occurrencies of the word we
just replaced, then we increment the number of occurrencies.

Our python implementation of this algorithm is slightly different from the
classic one described above, since we test the word against a list of stop words
and we added a hashmap to avoid useless
replacement. In our implementation, then, the following steps are performed:
\begin{enumerate}
	\item if the word is in the stop word lists, does nothing
	\item otherwise compute the hash of the given word
	\item increment the value associated to that hash
	\item check whether the value stored in the hash map is above the number of
		occurrencies of the less frequent term tracked by the SpaceSaving
		algorithm. If so, the replacement occurs, otherwise nothing happens.
\end{enumerate}
The output of this algorithm is a couple $(word, value)$ where the value is
computed as 
\begin{displaymath}
	value = occurrencies - error
\end{displaymath}
In order to reduce the noise we chose to use only couples where the value is
above a certain threshold.

\subsubsection*{Description of the methodology}
We have seen how we exploit the SpaceSaving algorithm to obtain the list of
frequent terms in a text and their
values. 

To accomplish our task, we ran the program described above on the set of tweets inside a
<<<<<<< HEAD
\emph{contradiction point}[CP], thus obtaining the list of frequent terms within the
contradiction; then, we read all the news which were published in the same time
period (a certain error between the publishing date and the contradiction
window) and for each of them we compute its score as
=======
\emph{contradiction pint} thus obtaining the list of frequent terms inside the
contradictions; then, we read all the news which were published during the
contradiction (a certain error between the publishing date and the contradiction
window might be taken in account) and for each of them we compute its score as
>>>>>>> 7a25539a6cb58d6ecc095f13a0f838184e4e830d
\begin{displaymath}
	newsScore = \sum_{w \in W} value(w)
\end{displaymath}
where $W$ is the list of words in the news and the value is null if that term is
not in the \emph{frequent term list}.

The news with the highest score is selected as the one causing the shift and its
first words are presented as output to the user.

This approach revealed to be very fast and from its formulation we can see that
it can be applied incrementally. As a drawback, we must say that it does not
take in account the background coming from the topic, hence it is not able to
discern whether a given frequent term is informative or not (e.g. running this
program on the Obama topic it is likely that the most
frequent terms will be ``President'', `Barack'', ``Obama'', ``USA'', hence a
news containing more repetitions of these words will have a high score, even if
those words are not informative at all)


\subsubsection*{LSI evaluation}
%\begin{table*}
%	\centering
%	\begin{tabularx}{\textwidth}{|l|X|}
%		\hline
%		Contr. point & Output \\
%		Cern1		&
%		Cern2		&
%		FortHood	&
%		HangOver	&
%		Lcross		&
%		Jackson1	&
%		Jackson2	&
%		SwineFlu	&
%	\end{tabularx}
%	\caption{Results achieved using LSI}
%	\label{tab:resultsLSI}
%\end{table*}


\subsubsection*{Tf/idf evaluation}
%\begin{table*}
%	\centering
%	\begin{tabularx}{\textwidth}{|l|X|}
%		\hline
%		Contr. point & Output \\
%		Cern1		&
%		Cern2		&
%		FortHood	&
%		HangOver	&
%		Lcross		&
%		Jackson1	&
%		Jackson2	&
%		SwineFlu	&
%	\end{tabularx}
%	\caption{Results achieved using Tf/idf}
%	\label{tab:resultsTFIDF}
%\end{table*}

\subsubsection*{Ngram Graph evaluation}
%\begin{table*}
%	\centering
%	\begin{tabularx}{\textwidth}{|l|X|}
%		\hline
%		Contr. point & Output \\
%		Cern1		&
%		Cern2		&
%		FortHood	&
%		HangOver	&
%		Lcross		&
%		Jackson1	&
%		Jackson2	&
%		SwineFlu	&
%	\end{tabularx}
%	\caption{Results achieved using ngram graphs}
%	\label{tab:resultsNGG}
%\end{table*}

