\subsection*{Experimental setup}
In order to perform the experimental evaluation we used a tool to automatically
detect sentiment shift in tweets coming from year 2009 and regarding several
topics. We also downloaded news from the \emph{New York Times} and \emph{ABC
Australia} on the same topic and spanning on the same period of the tweets.

It's important to remember that tweet have different language structure that need to be normalized, so for cleaning the tweet text the following operations were performed before the computation of the different methodologies:
\begin{itemize}
	\item URLs removal from tweets using a regular expression.
	\item conversion from Unicode to ASCII.
\end{itemize}

After that, we manually labelled each contradiction point with the event which
caused it. The topics, contradiction points and events used for the experiments
are showed in table \ref{tab:setup}

\begin{table*}
	\centering
	\begin{tabularx}{\textwidth}{|l|l|X|}
	\hline
	Contr. point 	& Contr. window 			& Event \\
	\hline 
	Cern1			& 2009-10-29 - 2009-11-08 	& On November 3rd a bird drops a
piece of bread which cause LHC overheating. \\
	Cern2			& 2009-11-24 - 2009-12-04	& On November 30th LHC accelerates protons to an
energy of 1.18 TeV, becoming the world most powerful energy particle
accelerator\\
	FortHood 		& 2009-11-02 - 2009-11-06	& On November 5th a US marine
kills 13 people\\
	HangOver		& 2009-06-21 - 2009-06-27	& undetermined or movie released
	on June 5th\\
	Lcross			& 2009-10-31 - 2009-11-08	& Preliminary findings from
	Lcross. Others announced\\
	Jackson1		& 2009-06-19 - 2009-06-25	& On June 25th Michael Jackson
dies\\
	Jackson2		& 2009-08-23 - 2009-08-29	& On august 28th another popular
musician, Adam Goldstein, dies. The day after, August 29th is Jackson's
birthday\\
	SwineFlu		& 2009-06-19 - 2009-06-23	& The swine flu is recognized as
	a pandemic\\
	\hline
	\end{tabularx}
	\caption{Contradiction points used for experimental evaluation}
	\label{tab:setup}
\end{table*}

\subsection*{Experiments}
\subsubsection*{SpaceSaving evaluation}
The results achieved using the emph{SpaceSaving} methodology with a maximum
error of three days on the contradiction window are shown in table
\ref{tab:resultsSS}.

\begin{table*}
	\centering
	\begin{tabularx}{\textwidth}{|l|X|}
		\hline
		Contr. point & Output \\
		Cern1		& Peckish bird crashes atom smasher. A peckish bird has
		briefly knocked out part of the world's biggest atom smasher by causing
		a chain reaction with a pi...\\
		Cern2		& Atom-smasher sets energy recordThe world's biggest
		atom-smasher has set a world record by accelerating to energy levels
		that had never been previously...\\
		FortHood	&
		HangOver	&
		Lcross		&
		Jackson1	&
		Jackson2	&
		SwineFlu	&
	\end{tabularx}
	\caption{Results achieved using SpaceSaving}
	\label{tab:resultsSS}
\end{table*}

\subsubsection*{LSI evaluation}
\begin{table*}
	\centering
	\begin{tabularx}{\textwidth}{|l|X|}
		\hline
		Contr. point & Output \\
		Cern1		&
		Cern2		&
		FortHood	&
		HangOver	&
		Lcross		&
		Jackson1	&
		Jackson2	&
		SwineFlu	&
	\end{tabularx}
	\caption{Results achieved using LSI}
	\label{tab:resultsLSI}
\end{table*}


\subsubsection*{Tf/idf evaluation}
\begin{table*}
	\centering
	\begin{tabularx}{\textwidth}{|l|X|}
		\hline
		Contr. point & Output \\
		Cern1		&
		Cern2		&
		FortHood	&
		HangOver	&
		Lcross		&
		Jackson1	&
		Jackson2	&
		SwineFlu	&
	\end{tabularx}
	\caption{Results achieved using Tf/idf}
	\label{tab:resultsTFIDF}
\end{table*}

\subsubsection*{Ngram Graph evaluation}
\begin{table*}
	\centering
	\begin{tabularx}{\textwidth}{|l|X|}
		\hline
		Contr. point & Output \\
		Cern1		&
		Cern2		&
		FortHood	&
		HangOver	&
		Lcross		&
		Jackson1	&
		Jackson2	&
		SwineFlu	&
	\end{tabularx}
	\caption{Results achieved using ngram graphs}
	\label{tab:resultsNGG}
\end{table*}

